% CUSTOM BEGIN

% Date to print on the document.
\newcommand{\npHandoverDate}{10.03.2017}

% If you use this command as usual after \appendix it only works only, if there is another "real" appendix, that means an appendix not created using a \addtocontents command.  If there is no real appendix, you must include this command in a \BeforeClosingMainAux (see below)
% Parameter 1: Type of entry to add, either chapter or part
% Parameter 2: Name of the entry to add
\newcommand{\addcontentslineWithoutPagenumber}[2]{
\stepcounter{#1}
\addtocontents{toc}{\protect\originalContentsline{#1}{\ifstrequal{#1}{chapter}{\numberline{\thechapter}}{}#2}{}}
}
% chapter without number but with toc entry
\newcommand{\chapterUnnumbered}[1]{
    \chapter*{#1}
    \addcontentsline{toc}{chapter}{#1}
}

% Uncomment this, if your appendix has only ''virtual'' entries.
%% Placing the addcontentsline commands here allows having an appendix, that only consists of "virtual" entries, that means, entries added via addcontents* commands. If there are no real entries, the usual approach (placing the entries after \appendix) does not work.
%\BeforeClosingMainAux{
%	% Add `Appendix` to TOC
%	\addcontentslineWithoutPagenumber{part}{\appendixname~(auf dem Datenträger)}
%	%\addtocontents{toc}{\protect\originalContentsline{part}{\appendixname}{}}
%	
%	% must be _input_, otherwise the TOC entry is at the wrong place
%	% !TeX encoding=UTF-8
% !TeX spellcheck = de-DE
% !TeX root = ../LaTeXTemplate.tex

%
% add files for appendix chapter here
% !TeX encoding=UTF-8
% !TeX spellcheck = de-DE
% !TeX root = ../LaTeXTemplate.tex

\chapter{First chapter of appendix}
\label{chap:Appendix:A}

\section{Parameters}
\label{sec:Appendix:Parameter}


%}

\usepackage{blindtext}

% Source: http://tex.stackexchange.com/a/341195/61241

% Trys to wrap the text inside \inlinecodeWrap if it overflows the line. If
% a single word is too long and can not be wrapped it will overflow.
% In such cases you should use \inlinecodeNoWrap instead.
% Rounded corners.

% NOTE: You to compile it twice.

%\usepackage{tikz}
\usepackage{soul}
\usetikzlibrary{calc}

\definecolor{veryDarkGray}{rgb}{0.3,0.3,0.3}

\makeatletter

\newcommand{\defhighlighter}[3][]{%
  \tikzset{every highlighter/.style={color=#2, fill opacity=#3, #1}}%
}

\defhighlighter[rounded corners =1mm]{veryDarkGray}{.15}

\newcommand{\highlight@DoHighlight}{
  \fill [outer sep = -15pt, inner sep = 0pt
        , every highlighter, this highlighter ]
        ($(begin highlight)+(0,8pt)$) rectangle ($(end highlight)+(0,-3pt)$) ;
}

\newcommand{\highlight@BeginHighlight}{
  \coordinate (begin highlight) at (0,0) ;
}

\newcommand{\highlight@EndHighlight}{
  \coordinate (end highlight) at (0,0) ;
}

\newdimen\highlight@previous
\newdimen\highlight@current

\DeclareRobustCommand*\highlight[1][]{%
  \tikzset{this highlighter/.style={#1}}%
  \SOUL@setup
  %
  \def\SOUL@preamble{%
    \begin{tikzpicture}[overlay, remember picture]
      \highlight@BeginHighlight
      \highlight@EndHighlight
    \end{tikzpicture}%
  }%
  %
  \def\SOUL@postamble{%
    \begin{tikzpicture}[overlay, remember picture]
      \highlight@EndHighlight
      \highlight@DoHighlight
    \end{tikzpicture}%
  }%
  %
  \def\SOUL@everyhyphen{%
    \discretionary{%
      \SOUL@setkern\SOUL@hyphkern
      \SOUL@sethyphenchar
      \tikz[overlay, remember picture] \highlight@EndHighlight ;%
    }{%
    }{%
      \SOUL@setkern\SOUL@charkern
    }%
  }%
  %
  \def\SOUL@everyexhyphen##1{%
    \SOUL@setkern\SOUL@hyphkern
    \hbox{##1}%
    \discretionary{%
      \tikz[overlay, remember picture] \highlight@EndHighlight ;%
    }{%
    }{%
      \SOUL@setkern\SOUL@charkern
    }%
  }%
  %
  \def\SOUL@everysyllable{%
    \begin{tikzpicture}[overlay, remember picture]
      \path let \p0 = (begin highlight), \p1 = (0,0) in \pgfextra
        \global\highlight@previous=\y0
        \global\highlight@current =\y1
      \endpgfextra (0,0) ;
      \ifdim\highlight@current < \highlight@previous
        \highlight@DoHighlight
        \highlight@BeginHighlight
      \fi
    \end{tikzpicture}%
    \ttfamily \the\SOUL@syllable
    \tikz[overlay, remember picture] \highlight@EndHighlight ;%
  }%
  \SOUL@
}
\makeatother

\newcommand{\inlinecodeWrap}[1]{\highlight{\small\ttfamily#1}}
% Source: http://tex.stackexchange.com/a/341192/61241

% Wraps the complete text in \inlinecodeNoWrap to the next line if it does not fit in the current line.
% Round corners, but no borders.

%\usepackage{tikz}

% NOTE: If it does not work as expected you may uncomment the lines below and 
% try to wrap the code using \inlinecodeNoWrap in the InlineCodeNoWrap environment. 
% Example: 
% \begin{InlineCodeNoWrap}
% 		\inlinecodeNoWrap{MyCode}
% \end{InlineCodeNoWrap}

%\newenvironment{InlineCodeNoWrap}
%{\par
%\hyphenpenalty=10000
%\exhyphenpenalty=10000
%}
%{\par}

\newcommand\lword[1]{\leavevmode\nobreak\hskip0pt plus\linewidth\penalty50\hskip0pt plus-\linewidth\nobreak #1}

\definecolor{veryLightGray}{rgb}{0.9,0.9,0.9}

\newcommand\roundInlineBox[2][]{\tikz[overlay]% Comment to prevent whitespace.
\node[fill=veryLightGray,inner sep=1pt, anchor=text, rectangle, rounded corners=1mm,#1] {#2};% Comment to prevent whitespace.
\phantom{#2}}

\newcommand{\inlinecodeNoWrap}[1]{\lword{\roundInlineBox{\small\ttfamily{#1}}}}

% for all languages
\lstset{
	  basicstyle = {\singlespacing\fontsize{10pt}{12pt}\ttfamily},
	  aboveskip = 0pt,
	  belowskip = 0pt,
   extendedchars=true,
   showstringspaces=false,
   showspaces=false,
   numbers=left,
   numberstyle=\footnotesize,
   numbersep=9pt,
   tabsize=2,
   breaklines=true,
   showtabs=false,
   captionpos=b
}

% define \term and (to only emphasize first occurence of a term, uncomment the commented lines)
\def\term#1{%
%  \ifcsname TERM#1\endcsname%
%    #1%
%  \else%
    \emph{#1}%
%    \expandafter\gdef\csname TERM#1\endcsname{}%
%  \fi
}
\makeglossaries 
% emphasize first acronym occurence
\defglsdisplayfirst[acronym]{\emph{#1#4}}

% You should not use \gls{} in headlines created with \chapter, \section, ...
% because it will only be formatted correctly in the TOC entry, but not in the actual headline
% and if you use back references from the list of acronyms to the pages referring to this gls entry
% the TOC page would falsely be listed for an gls entry used in a headline.
%
% Consider that these commands always print the full gls entry and do not increment
% the first occurence counter.

% Format a gls entry in a headline as if it was in the flow text, but
% avoid the disadvantages when using \gls{} in a headline.
\let\glsAsPartOfHeadline\glsentryfull

% The command \glsOnlyHeadline will apply a dedicated formatting for gls
% and should be used in headlines, which only contain the gls entry, that means,
% no text surrounding the gls entry. 
% Example: \section{\gls{KEY}}, but NOT \section{Some text before the \gls{KEY} and some text after it}
\newcommand{\glsOnlyHeadline}[1]{% do NOT remove this comment, it prevents a leading whitespace
	\glsentryshort{#1} \textendash {} \glsentrylong{#1}
}

\renewcommand{\glossarysection}[2][\theglstoctitle]{%
  \def\theglstoctitle{#2}%
  \chapterUnnumbered{#2}
  \vspace{1em}
  \par\noindent
}

% for color schemes, a small box filled with the color and bordered lightgray
 \newcommand{\colorSchemeBox}{\tikz \filldraw [draw=lightgray, semithick] (0.1,0.1) rectangle (0.5,0.5);}

\onehalfspacing
\usepackage{chngcntr} 
\counterwithout{footnote}{chapter} % Enumerate footnotes across chapter, i.e., do not start with 1 in each chapter

\usepackage{seqsplit}
\usepackage{ifthen}

% Remove the comments, if you want to use a numeric bibstyle. 
% NOTE: You have to load biblatex like this: usepackage[style=numeric-comp]{biblatex}
% \textualcite[see][42]{key} will produce
% this cite: see [1], page 42. Other cite commands set the complete citation in brackets: [see 1, page 42].
\newbibmacro*{textualcite}{%
  \printtext[bibhyperref]{%
    \printfield{prefixnumber}%
    \printfield{labelnumber}%
    \ifbool{bbx:subentry}
      {\printfield{entrysetcount}}{}% Do not remove this comment to prevent unwanted whitespace.
}}
\DeclareCiteCommand{\textualcite}
  {\iffieldundef{prenote}%
    {}%
    {\usebibmacro{prenote}\addspace}%
    \bibopenbracket\setunit{}}%
  {\usebibmacro{citeindex}%
   \usebibmacro{textualcite}}%
  {\multicitedelim}%
  {\bibclosebracket%
    \usebibmacro{postnote}}

% Custom cite command to allow easy switching between, e.g., \textualcite and \cite
% If you use a cite command different from \textualcite you might have to change 
% the loading of biblatex like this: usepackage[style=alphabetic]{biblatex}
\DeclareDocumentCommand{\MyCite}{ O{default1} O{default2} m }{
\ifdef{\textualcite}{
	\textualcite[#1][#2]{#3}
}
{
	\cite[#1][#2]{#3}
}
}

% Makes sure, that the line number in lstinputlisting shows the line numbers corresponding to the whole file, even if you just print a part of the file. See http://tex.stackexchange.com/a/110195/61241
\makeatletter
\lst@Key{matchrangestart}{f}{\lstKV@SetIf{#1}\lst@ifmatchrangestart}
\def\lst@SkipToFirst{%
    \lst@ifmatchrangestart\c@lstnumber=\numexpr-1+\lst@firstline\fi
    \ifnum \lst@lineno<\lst@firstline
        \def\lst@next{\lst@BeginDropInput\lst@Pmode
        \lst@Let{13}\lst@MSkipToFirst
        \lst@Let{10}\lst@MSkipToFirst}%
        \expandafter\lst@next
    \else
        \expandafter\lst@BOLGobble
    \fi}
\makeatother

\lstset{numbers=left,matchrangestart=t}

% Allows page numbers in \cite commands to contain a "-" (usually the dash would be 
% interpreted as a page range). E. g.: \citep[12\pnhyph 2]{myCiteId} cites page "12-2"
\def\pnhyph{-}
\NumCheckSetup{\let\pnhyph\empty}

% force breaking lines in bibliography
\setcounter{biburllcpenalty}{7000}
\setcounter{biburlucpenalty}{8000}

% Uncomment this to compile without justification, hyphenation, citations and listings for spell checking.
%\renewcommand{\lstinputlisting}[2][]{}
%\renewcommand{\inlinecodeWrap}[1]{}
%\renewcommand{\inlinecodeNoWrap}[1]{}
%\DeclareCiteCommand{\textualcite}{}{Quelle}{}{}
%\DeclareDocumentCommand{\MyCite}{ O{default1} O{default2} m }{Quelle}
%\renewcommand{\Cref}[1]{Abschnitt 1}
%\renewcommand{\footnote}{}
%\DeclareDocumentCommand{\cite}{ O{} O{} m }{Quelle}
%\IfElseDefined{chapter}{
%   \ohead[]{} % header : nothing
%   \ihead[]{} 
%   \ofoot[]{} 	% footer: nothing
%   \setheadsepline{0pt}[] % disables the line below the header
%} {}
% disable hyphenation
%\tolerance=1
%\emergencystretch=\maxdimen
%\hyphenpenalty=10000
%\hbadness=10000
%\raggedright % set left aligned instead of justified
% CUSTOM END