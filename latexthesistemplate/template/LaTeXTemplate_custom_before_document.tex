% custom begin
% If you use this command as usual after \appendix it only works only, if there is another "real" appendix, that means an appendix not created using a \addtocontents command.  If there is no real appendix, you must include this command in a \BeforeClosingMainAux (see below)
% Parameter 1: Type of entry to add, either chapter or part
% Parameter 2: Name of the entry to add
\newcommand{\addcontentslineWithoutPagenumber}[2]{
\stepcounter{#1}
\addtocontents{toc}{\protect\originalContentsline{#1}{\ifstrequal{#1}{chapter}{\numberline{\thechapter}}{}#2}{}}
}
% chapter without number but with toc entry
\newcommand{\chapterUnnumbered}[1]{
    \chapter*{#1}
    \addcontentsline{toc}{chapter}{#1}
}

% Uncomment this, if your appendix has only ''virtual'' entries.
%% Placing the addcontentsline commands here allows having an appendix, that only consists of "virtual" entries, that means, entries added via addcontents* commands. If there are no real entries, the usual approach (placing the entries after \appendix) does not work.
%\BeforeClosingMainAux{
%	% Add `Appendix` to TOC
%	\addcontentslineWithoutPagenumber{part}{\appendixname~(auf dem Datenträger)}
%	%\addtocontents{toc}{\protect\originalContentsline{part}{\appendixname}{}}
%	
%	% must be _input_, otherwise the TOC entry is at the wrong place
%	% !TeX encoding=UTF-8
% !TeX spellcheck = de-DE
% !TeX root = ../LaTeXTemplate.tex

%
% add files for appendix chapter here
% !TeX encoding=UTF-8
% !TeX spellcheck = de-DE
% !TeX root = ../LaTeXTemplate.tex

\chapter{First chapter of appendix}
\label{chap:Appendix:A}

\section{Parameters}
\label{sec:Appendix:Parameter}


%}

\usepackage{blindtext}
% for code inside the text
\definecolor{veryLightGray}{rgb}{0.9,0.9,0.9}
\definecolor{verydarkGray}{rgb}{0.1,0.1,0.1}
\newcommand\roundInlineBox[2][]{\tikz[overlay]\node[fill=veryLightGray,inner sep=1pt, anchor=text, rectangle, rounded corners=1mm,#1] {#2};\phantom{#2}}
\newcommand{\inlinecode}[1]{\roundInlineBox{\small\ttfamily\color{verydarkGray}{#1}}}
% JS for listings
\definecolor{darkgray}{rgb}{.4,.4,.4}
\definecolor{purple}{rgb}{0.65, 0.12, 0.82}

%\definecolor{colKeys}{rgb}{0,0,1}
%\definecolor{colIdentifier}{rgb}{1,0,0}
%\definecolor{colComments}{rgb}{0,0.7,0.4}
%\definecolor{colString}{rgb}{0,0.5,0}

% for all languages
\lstset{
	  basicstyle = {\singlespacing\fontsize{10pt}{12pt}\ttfamily},
	  aboveskip = 0pt,
	  belowskip = 0pt,
   extendedchars=true,
   showstringspaces=false,
   showspaces=false,
   numbers=left,
   numberstyle=\footnotesize,
   numbersep=9pt,
   tabsize=2,
   breaklines=true,
   showtabs=false,
   captionpos=b
}

% define \term and (to only emphasize first occurence of a term, uncomment the commented lines)
\def\term#1{%
%  \ifcsname TERM#1\endcsname%
%    #1%
%  \else%
    \emph{#1}%
%    \expandafter\gdef\csname TERM#1\endcsname{}%
%  \fi
}
\makeglossaries 
% emphasize first acronym occurence
\defglsdisplayfirst[acronym]{\emph{#1#4}}

\newcommand{\glsInHeadline}[1]{% do NOT remove this comment, it prevents a leading whitespace
	\glsentryshort{#1} \textendash {} \glsentrylong{#1}
}

\renewcommand{\glossarysection}[2][\theglstoctitle]{%
  \def\theglstoctitle{#2}%
  \chapterUnnumbered{#2}
  \vspace{1em}
  \par\noindent
}


% for color schemes, a small box filled with the color and bordered lightgray
 \newcommand{\colorSchemeBox}{\tikz \filldraw [draw=lightgray, semithick] (0.1,0.1) rectangle (0.5,0.5);}

% custom end


%\lstset{
%  language        = php,
%  keywordstyle    = \color{blue},
%  stringstyle     = \color{red},
%  identifierstyle = \color{green},
%  commentstyle    = \color{gray},
%  emph            =[1]{php},
%  emphstyle       =[1]\color{black},
%  emph            =[2]{if,and,or,else},
%  emphstyle       =[2]\color{yellow}
%}


% \input{preamble/style-geometry.tex}
\onehalfspacing
\usepackage{chngcntr} 
\counterwithout{footnote}{chapter}
\usepackage{remreset}
% CUSTOM BEGIN
\usepackage{seqsplit}
\usepackage{ifthen}
% CUSTOM END
%CUSTOM BEGIN

\newbibmacro*{textualcite}{%
  \printtext[bibhyperref]{%
    \printfield{prefixnumber}%
    \printfield{labelnumber}%
    \ifbool{bbx:subentry}
      {\printfield{entrysetcount}}
      {}}}
\DeclareCiteCommand{\textualcite}%[\mkbibbrackets]
  {\iffieldundef{prenote}%
    {}%
    {\usebibmacro{prenote}\addspace}%
    \bibopenbracket\setunit{}}%
  {\usebibmacro{citeindex}%
   \usebibmacro{textualcite}}%
  {\multicitedelim}%
  {\bibclosebracket%
    \usebibmacro{postnote}}



% Makes sure, that the line number in lstinputlisting shows the line numbers corresponding to the whole file, even if you just print a part of the file. See http://tex.stackexchange.com/a/110195/61241
\makeatletter
\lst@Key{matchrangestart}{f}{\lstKV@SetIf{#1}\lst@ifmatchrangestart}
\def\lst@SkipToFirst{%
    \lst@ifmatchrangestart\c@lstnumber=\numexpr-1+\lst@firstline\fi
    \ifnum \lst@lineno<\lst@firstline
        \def\lst@next{\lst@BeginDropInput\lst@Pmode
        \lst@Let{13}\lst@MSkipToFirst
        \lst@Let{10}\lst@MSkipToFirst}%
        \expandafter\lst@next
    \else
        \expandafter\lst@BOLGobble
    \fi}
\makeatother

\lstset{numbers=left,matchrangestart=t}

% spacing of section, ... titles
%\titlespacing{\section}{0pt}{*1}{*0.5}
%\titlespacing{\subsection}{0pt}{*1}{*0.5}
%\titlespacing{\subsubsection}{0pt}{*1}{*0.5}

% Allows page numbers in \cite commands to contain a "-" (usually the dash would be 
% interpreted as a page range). E. g.: \citep[12\pnhyph 2]{myCiteId} cites page "12-2"
\def\pnhyph{-}
\NumCheckSetup{\let\pnhyph\empty}

% force breaking lines in bibliography
\setcounter{biburllcpenalty}{7000}
\setcounter{biburlucpenalty}{8000}

% Uncomment this to compile without citations and listings for spell checking.
%	\renewcommand{\lstinputlisting}[2][]{}
%	\renewcommand{\inlinecode}[1]{}
%	\ifthenelse{\boolean{isSpellCheck}}{\DeclareCiteCommand{\textualcite}{}{Quelle}{}{}}{}
%	\renewcommand{\Cref}[1]{Abschnitt 1}

%

% Uncomment to set hyperref link color to text color for printing
%\UseDefinition{Target}{Print}

% CUSTOM END